\documentclass{article}
\usepackage[top=2.5cm, bottom=2.5cm, left=2cm, right=2cm]{geometry}
\usepackage{amsmath} 
\usepackage{amssymb}
\usepackage{mathrsfs}
\usepackage{amscd}
\usepackage{graphicx} 
\usepackage{subfig} 
\usepackage{tabularx}
\usepackage{indentfirst} 
\usepackage{array}
\usepackage{longtable}
\usepackage{multirow}
\usepackage{bm}
\usepackage{cite}
\usepackage{enumerate}

\author{Wang Dongying}
\title{Applications of Fermi Golden Rule}
\begin{document}
\maketitle

	\section{Transition Probabilities and Fermi Golden Rule}

		The famous Fermi Golden Rule describe the transition rate which depends on the strength of the coupling between the initial and fibal state of a system and upon the number of ways the transition can happen.

		\begin{equation}
		\begin{aligned}
			\mathscr{W}(t) &= \frac{d\mathscr{P}(t)}{dt}\\
						   &= \frac{2\pi}{\hbar}W_{fi}^{2}\rho(E_{fi})
		\end{aligned}
		\end{equation}

		where the $W_{fi}$ is matrix element for the interaction, $\rho(E_{fi})$ is density of the final state.\\

		Lifetime: $\tau = 1/\mathscr{W}$.

	\section{Scattering and Decays from Fermi Golden Rule}

		\paragraph{Fermi's Theory of nuclear $\beta$-decay}

			$n \to p + e^{-} + \bar{\nu}\ \ \ \ \ \ \ \ \ \ (Z,N) \to (Z+1, Z-1) + e^{-} + \bar{\nu}$

			The simplest form for thr matrix element describing nuclear $\beta$-decay is given by Fermi's ansatz:
			\begin{equation}
				W_{km} = \frac{G_{z}M}{V}
			\end{equation}
			where $G_{z}$ is Fermi constant, $M$ is overlap of the initial/final nuclear wave function and $V$ is the normalization volume. The density of possible states $dn/dE_{0} = \rho(E_{0})$. Typical value of $E_{0}$ is in the MeV range.\\
			Main equations:
			$$E_{0} = E + cq\ \ \ \ \ \ \ \ \ 0 = P + q + p$$
			where $E$ is electron energy, $q$ is neutrino momentum, $P$ is nuclear momentum and $p$ is electron mometum. The density of states is found from the product of the electron and neutrino phase space volumes.

			\begin{equation}
				dn = \frac{Vd^{3}\bm{p}}{(2\pi\hbar)^{3}}\cdot\frac{Vd^{3}\bm{q}}{(2\pi\hbar)^{3}} = \frac{1}{4\pi^{4}\hbar^{6}c^{3}}p^{2}(E - E_{0})^{2}dpdE
			\end{equation}

			Back to the expression of transition rate,

			\begin{equation}
				d\mathscr{W} = \frac{G_{f}^{2}}{2\pi^{3}\hbar^{7}c^{3}}|M|^{2}p^{2}(E - E_{0})^{2}dp
			\end{equation}

			when the electron can be treated as being relastitic($E = cp$).

			\begin{equation}
				\int p^{2}(E - E_{0})^{2}dp = \frac{Q_{0}^{5}}{30c^{3}}
			\end{equation}

			where $Q_{0} = E_{0} - m_{e}c^{2}$. Then the transition rate can be calculated.

			\begin{equation}
				\mathscr{W} = \frac{G_{f}^{2}}{60\pi^{3}\hbar^{7}c^{6}}|M|^{2}Q_{0}^{5}
			\end{equation}

		\paragraph{cross section}

			\begin{equation}
				\mathscr{W} = d\sigma j_{i}
			\end{equation}

			where $\sigma$ is cross section and $j_{i}$ is the flux of initial particals.









\end{document}